% REMEMBER TO SET LANGUAGE!
\documentclass[a4paper,10pt,english]{article}
\usepackage[utf8]{inputenc}
\usepackage[norsk]{babel}
% Standard stuff
\usepackage{amsmath,graphicx,babel,varioref,verbatim,amsfonts}
% colors in text
\usepackage[usenames,dvipsnames,svgnames,table]{xcolor}
% Hyper refs
\usepackage[framestyle=none,framefit=yes,heightadjust=all,framearound=all]{floatrow}    
\floatsetup[figure]{style=Boxed,framearound=all}
\usepackage{fancyhdr}
\usepackage[colorlinks]{hyperref}

% Document formatting
\setlength{\parindent}{0mm}
\setlength{\parskip}{1.5mm}
%\setcounter{section}{-1} Hvis du vil ha 0 som første avsnitt
\numberwithin{figure}{section} 
\numberwithin{table}{section}
\numberwithin{equation}{section}

%Color scheme for listings
\usepackage{textcomp}
\definecolor{listinggray}{gray}{0.9}
\definecolor{lbcolor}{rgb}{0.9,0.9,0.9}

%Listings configuration
\usepackage{listings}
\lstset{
	backgroundcolor=\color{lbcolor},
	tabsize=4,
	rulecolor=,
	language=python,
        basicstyle=\scriptsize,
        upquote=true,
        aboveskip={1.5\baselineskip},
        columns=fixed,
	numbers=left,
        showstringspaces=false,
        extendedchars=true,
        breaklines=true,
        prebreak = \raisebox{0ex}[0ex][0ex]{\ensuremath{\hookleftarrow}},
        frame=single,
        showtabs=false,
        showspaces=false,
        showstringspaces=false,
        identifierstyle=\ttfamily,
        keywordstyle=\color[rgb]{0,0,1},
        commentstyle=\color[rgb]{0.133,0.545,0.133},
        stringstyle=\color[rgb]{0.627,0.126,0.941}
        }
        
%User settings
\newcounter{subproject}
\renewcommand{\thesubproject}{\alph{subproject}}
\newenvironment{subproj}{
\begin{description}
\item[\refstepcounter{subproject}(\thesubproject)]
}{\end{description}}
\newcommand{\HRule}{\rule{\linewidth}{0.5mm}}
\newcommand{\eqs}{\begin{equation}}
\newcommand{\eqf}{\end{equation}}

%Other
\usepackage{geometry}
\usepackage{booktabs}
\usepackage[boxed,linesnumbered,lined]{algorithm2e}

%Lettering instead of numbering in different layers
%\renewcommand{\labelenumi}{\alph{enumi}}
%\renewcommand{\thesubsection}{\alph{subsection}}

\title{Jacobi method}
\author{Vidar Skogvoll}

\begin{document}
%Header
\pagestyle{fancy}
\renewcommand{\sectionmark}[1]{\markright{#1}{}}

\pagestyle{fancy}
\renewcommand{\sectionmark}[1]{\markright{\thesection\ #1}}

\fancyhf{}
%Header
%\rhead{\fancyplain{}{ NAVN PÅ PROSJEKT}} % predefined ()
\lhead{\fancyplain{}{\rightmark }} % 1. sectionname, 1.1 subsection name etc
\cfoot{\fancyplain{}{\thepage}}

\maketitle


\section{Method}

We want to solve the equation 

\eqs A x  = \lambda x \eqf

Using 

\section{The frobenius norm}

\eqs || A ||_F = \sqrt{\sum_{i,j}a_{i,j}^2} \eqf

During the similarity transformation 

\eqs B = S^T A S \eqf

The frobenius norm is conserved.

For the $2x2$ case

\eqs a_{pp}^2 + a_{qq}^2 + 2a_{pq}^2 = b_{pp}^2 + b_{qq}^2 + 2 b_{pq} = b_{pp}^2 + b_{qq}^2 \eqf

Since they are symmetric. Now

\eqs off(B) = ||B||_F^2 - \sum_i^m b_{ii}^2 \eqf
\eqs = ||A||_F^2 - \sum_{i=1}^n a_{ii}^2 + 2 a{pp}^2 + a_{qq}^2 - b_{pp}^2 -b_{qq}^2
\eqf
\eqs = ||A||_F^2 - \sum_{i=1}^n a_{ii}^2 - 2 a_{pq}^2 \eqf
\eqs = off(A) - 2 a_{pq}^2 \eqf

And thus 

\eqs off(B) \leq off(A) \eqf

\section{Other}

\eqs b_{pq} = 0 = a_{pq}(c^2 - s^2) + (a_{pp} - a_{qq}) cs \eqf
\eqs \tau = \frac{a_{qq} - a_{pp}}{2a_{pq}} \eqf
\eqs \tau \theta = \frac{s}{c} \eqf

Which gives 

\eqs t^2 + 2 t \tau - 1 = 0 \eqf
\eqs t = -\tau \pm \sqrt{1 + \tau^2} \eqf

We have best convergence when the absolute value of $\theta$ is less than equal to 
$\pi/4$. 
We would then choose the smaller ofthe roots. 
Assume $\tau \geq 0$, then we're choosing 

\eqs t = -\tau + \sqrt{1 + \tau^2} \eqf

This may cause problems due to subtraction of very equal numbers. 
What we do is multiply the equation over and under with $(\tau + \sqrt{1 + \tau^2})$ 


\section{Estimate of floating point operations}
Referencing to the jacobi implemented by morten. 

\begin{table}[h!]
        \centering 
        \begin{tabular}{ll}
                \toprule
                Method & Flops \\
                \midrule
                off-diag & $n^2$ \\
                Rotation part (Withouth eigenvectors) & $6n$ (standard claim in txtbooks) \\ 
                \bottomrule 
        \end{tabular}
        \caption{Flops of the different jacobi.}
\end{table}
\end{document}